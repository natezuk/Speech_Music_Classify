\documentclass[11pt]{article}
\usepackage{parskip}
\usepackage{color}
\usepackage{graphicx}

\title{EEG-based classification of natural sounds reveals specialized responses to speech and music}
\author{Nathaniel Zuk, Emily Teoh, Edmund Lalor}

\begin{document}
\maketitle

\textbf{Abstract:}

\textbf{Introduction:}

\begin{itemize}
\item In previous work to study the processing of speech of music sounds, these sounds are resynthesized using vocoding or scrambling techniques in order to reduce the intelligibility of those sounds while maintaining simple aspects of their acoustic structure. Often, these studies show enhanced responses in EEG to the originals.  However, it is unclear whether this is a result of specialized processing in the brain or simply due to the coherence of features in the speech and music sounds.
\item In animals, there is a considerable amount of work showing that auditory cortex contains specialized responses to vocalization, and statistics of sounds most ecologically relevant for the animal.
\item Recent work in humans, using fMRI, has revealed regions of auditory cortex that are especially sensitive to speech and music sounds, compared to various other natural sounds.  The temporal properties of these responses, however, are less clear.
\item \textbf{Here we use classificiation-based analyses using EEG to identify specialized neural responses to speech and music sounds.  Furthermore, unlike other natural sounds that evoke unique temporal responses (such as impact sounds), these responses appear to respond to higher-level temporal organization of the speech and music sounds.  We also find temporal differences in the neural responses to these sounds: speech sounds can be classified using only evoked responses between 100-600 ms following stimulus onset, while evoked responses appear to contain information sporadically throughout the 2 s interval of the stimuli.  Our results highlight the specialization of the brain to speech and music sounds and reveal potential temporal differences in the processing of these sounds.}
\end{itemize}

\textbf{Methods:}

\textbf{Results:}

\textbf{Discussion:}



\end{document}